\documentclass[12pt]{extarticle}
\usepackage[utf8]{inputenc}
\usepackage{amsmath}
\usepackage{hyperref}
\usepackage[shortlabels]{enumitem}
\usepackage{tikz}
\usepackage{textcomp}

\addtolength{\textwidth}{1.0in}
\addtolength{\textheight}{0.75in}
\addtolength{\evensidemargin}{-0.75in}
\addtolength{\oddsidemargin}{-0.75in}
\addtolength{\topmargin}{-1.0in}

\title{Past Gauss Contest Problems}
\author{Eric Xiao}
\date{May 16, 2020}

\begin{document}

\maketitle

\section{Problems}
\begin{enumerate}
    \itemsep 2.0em
    \item {A rectangular sheet of paper measures 25 cm by 9 cm. What are the dimensions of a square sheet of paper with the same area?} %\\Answer: \fbox{15 cm by 15 cm}
    \item {Last year, Kiril's age was a multiple of 7. This year, Kiril's age is a multiple of 5. In how many years will Kiril be 26 years old?} %\\Answer: \fbox{11}
    \item {How many positive integers between 10 and 2016 are divisible by 3 and have all of their digits the same?} %\\Answer: \fbox{12}
    \item {At the Gaussland Olympics there are 480 student participants. Each student is participating in 4 different events. Each event has 20 students participating and is supervised by 1 adult coach. There are 16 adult coaches and each coach supervises the same number of events. How many events does each coach supervise?} %\\Answer: \fbox{6}
    \item {Line segments $PQ$ and $RS$ are parallel. Points $T, U, V$ are placed so that $\angle QTV$ = 30\textdegree, $\angle SUV$ = 40\textdegree, and $\angle TVU$ = x\textdegree, as shown. What is the value of $x$?
    \\ \includegraphics{Gauss2018-20.png}} %\\Answer: \fbox{$70$}
    \item {Mike and Alain play a game in which each player is equally likely to win. The first player to win three games becomes the champion, and no further games are played. If Mike has won the first game, what is the probability that Mike becomes the champion?} %\\Answer: \fbox{$\frac{11}{16}$}
    \item {A lattice point is a point $(x, y)$, with $x$ and $y$ both integers. For example, $(2, 3)$ is a lattice point but $(4, \frac{1}{3})$ is not. In the diagram, how many lattice points lie on the perimeter of the triangle?
    \\ \includegraphics{Gauss2007-24.png}} %\\Answer: \fbox{20}
    \item {The alternating sum of the digits of 63,195 is $6-3+1-9+5=0$. In general, the alternating sum of the digits of a positive integer is found by taking its leftmost digit, subtracting the next digit to the right, adding the next digit to the right, then subtracting, and so on. A positive integer is divisible by 11 exactly when the alternating sum of its digits is divisible by 11. For example, 63,195 is divisible by 11 since the alternating sum of its digits is equal to 0, and 0 is divisible by 11. Similarly, 92,807 is divisible by 11 since the alternating sum of its digits is 22, but 60,432 is not divisible by 11 since the alternating sum of its digits is 9. Lynne forms a 7-digit integer by arranging the digits $1,2,3,4,5,6,7$ in random order. What is the probability that the integer is divisible by 11?} %\\Answer: \fbox{$\frac{4}{35}$}
\end{enumerate}

\end{document}

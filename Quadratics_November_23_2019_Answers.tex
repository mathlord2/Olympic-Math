\documentclass[14pt]{extarticle}
\usepackage[utf8]{inputenc}
\usepackage{hyperref}

\addtolength{\textwidth}{1.0in}
\addtolength{\textheight}{0.75in}
\addtolength{\evensidemargin}{-0.75in}
\addtolength{\oddsidemargin}{-0.75in}
\addtolength{\topmargin}{-1.0in}

\title{\textbf{More Quadratic Polynomials}}
\author{Eric Xiao}
\date{October 19, 2019}

\begin{document}

\maketitle

\section{Vertex Form}
{Express each polynomial below in vertex form:}
\begin{enumerate}
    \item {$x^2 + 4x + 5$ = \fbox{$(x + 2)^2 + 1$}}
    \item {$-x^2 - 6x + 3$ = \fbox{$-(x + 3)^2 + 12$}}
    \item {$2x^2 - 7$ = \fbox{$2x^2 - 7$} (Wow, the question is the answer)}
    \item {$4x^2 - 8x + 17$ = \fbox{$4(x - 1)^2 + 13$}}
    \item {$-7x^2 + 21x - 14$ = \fbox{$-7(x - \frac{3}{2})^2 + \frac{7}{4}$}}
    \item {$8x^2 - 30x + \frac{1}{2}$ = \fbox{$8(x - \frac{15}{8})^2 - \frac{221}{8}$}}
\end{enumerate}

\section{Solving Equations in Vertex Form}
{Solve for the variable in each equation below:}
\begin{enumerate}
    \item {$(x - 3)^2 - 9 = 0$ \fbox{$x = 0, 6$}}
    \item {$(z - 5)^2 - 49 = 0$ \fbox{$z = -2, 12$}}
    \item {$2(a + 11)^2 - 10 = 22$ \fbox{$a = -15, -7$}}
    \item {$(k + 9)^2 - 29 = 1$ \fbox{$k = -9 - \sqrt{30}, -9 + \sqrt{30}$}}
    \item {$3(p + \frac{2}{3})^2 - 27 = 0$ \fbox{$p = -\frac{11}{3}, \frac{7}{3}$}}
    \item {$4(y - \frac{3}{4})^2 - 13 = 0$ \fbox{$y = \frac{3 - 2\sqrt{13}}{4}, \frac{3 + 2\sqrt{13}}{4}$}}
\end{enumerate}

\section{Solving Polynomial Equations}
{Solve by factoring or using the quadratic formula:}
\begin{enumerate}
    \item {$x^2 + 6x + 9 = 0$ \fbox{Factoring: $x = -3$}}
    \item {$2x^2 - 5x - 3 = 0$ \fbox{Factoring: $x = -\frac{1}{2}, 3$}}
    \item {$x^2 - 9x - 3 = 2$ \fbox{Quadratic Formula: $x = \frac{9 - \sqrt{101}}{2}, \frac{9 + \sqrt{101}}{2}$}}
    \item {$4x^2 + 15x + 7 = 0$ \fbox{Quadratic Formula: $x = \frac{-15 - \sqrt{113}}{8}, \frac{-15 + \sqrt{113}}{8}$}}
    \item {$-2x^2 - 2x + 19 = 11$ \fbox{Quadratic Formula: $x = \frac{-1 - \sqrt{17}}{2}, \frac{-1 + \sqrt{17}}{2}$}}
    \item {$-10x^2 + 41x - 24 = -3$ \fbox{Factoring: $x = \frac{3}{5}, \frac{7}{2}$}}
\end{enumerate}

\section{Word Problems}
\begin{enumerate}
    \itemsep5.0em
    \item {Convert this equation into vertex form to get its vertex. The equation in vertex form is $y = 3(x - 1)^2 - 3$. Therefore, the equation's vertex is at \fbox{(1, -3)}.}
    \item {This equation in vertex form is $y = -4(x - \frac{5}{2})^2 + 8$. Its vertex would be at ($\frac{5}{2}$, 8), so the y-value of the vertex is \fbox{8}. If you were to graph this equation, it would curve downwards from both sides of the vertex, since the coefficient in front of the brackets is -4, a negative value. Hence, this equation will never return a y-value greater than that of the vertex, which means that this y-value is a \fbox{maximum}.}
    \item {Let's assume that the graph has a vertex form of $y = a(x - p)^2 + q$ for some real numbers $a$, $p$, and $q$. Since the parabola's vertex is at (-3, 4), the values of $p$ and $q$ are -3, 4, respectively. So far, this gives us an equation of $y = a(x + 3)^2 + 4$. We are also given that the parabola intersect the point (-1, -4). To solve for $a$, we will substitute this point into the equation to obtain $-4 = a(-1 + 3)^2 + 4$. Solving this results in $a = -2$. Therefore, our equation in vertex form is $y = -2(x + 3)^2 + 4$, which expands to \fbox{$y = -2x^2 - 12x - 14$}.}
    \item {Let $x$ represent the length of the triangle's base. Since the triangle's height is 2 times longer than the base, we can express the height as $2x$. The triangle's area would be $\frac{1}{2}$ of its base multiplied by its height, which becomes $\frac{x(2x)}{2}$, simplifying to $x^2$. The rectangle has the same height as the triangle, $2x$, so its area is its base multiplied by its height, which is $5(2x)$, simplifying to $10x$. Since the sum of these areas is 96, we obtain the equation $x^2 + 10x = 96$. This factors as $(x + 16)(x - 6) = 0$, so we get x-values of -16 and 6. However, since the triangle's base length has to be a positive value, $x$ cannot equal -16, so the only value it can take is \fbox{6 cm}.}
    \item {The reason that our number system is called "base 10" is because we can write each number as a sum of powers of 10. For instance, the number 473 can be rewritten as 400 + 70 + 3, resulting in 4 x 10$^2$ + 7 x 10$^1$ + 3 x 10$^0$ (remember scientific notation?). Note that the first power of 10 has an exponent equal to the number of digits in 473 minus one, while the exponents for the next few powers decrease by 1, and the coefficients of each of these powers represent the digits of 473. Similarly, the number 473 in some base $n$ can be rewritten as 4 x $n^2$ + 7 x $n^1$ + 3 x $n^0$, and when we solve for this value, it will actually be in base 10. Using this knowledge, we can let the Marth's number system be in some base $x$ and find an equation to solve for $x$. We are given that the Marth number 111 is equal to 21 in base 10, so the equation that represents this is 1 x $x^2$ + 1 x $x^1$ + 1 x $x^0$ = 21. Simplify this to $x^2 + x + 1 = 21$, and now this looks like a quadratic equation (yay!). Factoring this equation gives us $(x + 5)(x - 4) = 0$, so $x$ = -5 or $x$ = 4. However, we can't have a number system with a base of a negative number, hence the inhabitants of Marth write their numbers in base 4. Therefore, the Marth number 2021 would equal 2 x $4^3$ + 0 x $4^2$ + 2 x $4^1$ + 1 x $4^0$, which simplifies to $128 + 0 + 8 + 1$, or \fbox{137}.}
\end{enumerate}

\end{document}

\documentclass{article}
\usepackage[utf8]{inputenc}
\usepackage{hyperref}

\title{\textbf{Exponents and Polynomial Equations - Answers}}
\author{Eric Xiao}
\date{October 19, 2019}

\addtolength{\textwidth}{1.0in}
\addtolength{\textheight}{1.00in}
\addtolength{\evensidemargin}{-0.75in}
\addtolength{\oddsidemargin}{-0.75in}
\addtolength{\topmargin}{-1.0in}

\begin{document}

\maketitle

{All final answers will have a box around them.}

\section{Solving Exponents}
\begin{enumerate}
    \itemsep1.0em
    \item {$2^3$ = \fbox{$8$}}
    
    \item {$10^7$ = \fbox{$10000000$}}
    
    \item {$-6^2$ = \fbox{$-36$}}
    
    \item {$8^{-2}$ = \fbox{$\frac{1}{64}$}}
    
    \item {$480349^0$ = \fbox{$1$}}
    
    \item {$(-4)^{-3}$ = \fbox{$-\frac{1}{64}$}}
    
    \item {$2^3$ x $2^{-2}$ = $2^1$ = \fbox{2}}
    
    \item {$\sqrt{64}$ = \fbox{8}}
    
    \item {$(-8)^\frac{1}{3}$ = \fbox{-2}}
\end{enumerate}

\section{Scientific Notation}
\begin{enumerate}
    \item {187 = \fbox{1.87 x $10^2$}}
    \item {5010000 = \fbox{5.01 x $10^6$}}
    \item {0.065 = \fbox{6.5 x $10^{-2}$}}
    \item {$\frac{1}{2}$ = \fbox{5 x $10^{-1}$}}
    \item {6 = \fbox{6 x $10^0$}}
\end{enumerate}

\section{Simplifying Exponents}
\begin{enumerate}
    \itemsep1.0em
    \item {$2x^2$ + $4x^2$ = \fbox{$6x^2$}}
    
    \item {$y^8$ x $y^{-9}$ = $y^{-1}$ = \fbox{$\frac{1}{y}$}}
    
    \item {$(a^2)^3$ = \fbox{$a^6$}}
    
    \item {$\frac{2x^2}{x^5}$ = $2x^{-3}$ = \fbox{$\frac{2}{x^3}$}}
    
    \item {$(3n)^{-3}$ x $(3n)^5$ = $(3n)^2$ = \fbox{$9n^2$}}
    
    \item {$z^{-1}$ x $z^\frac{1}{2}$ = $z^{-\frac{1}{2}}$ = \fbox{$\frac{1}{z^\frac{1}{2}}$} or \fbox{$\frac{1}{\sqrt{z}}$}}
    
    \item {$\frac{p^3}{p^\frac{4}{5}}$ = \fbox{$p^\frac{11}{5}$}}
    
\end{enumerate}

\section{Word and Thinking Problems}
\begin{enumerate}
    \itemsep1.5em
    
    \item {Let $n$ represent the original number. This gives us an equation of $n^3 - 8 = 56$. Bring the 8 over to the right side to obtain $n^3 = 64$. Finally, cube-root both sides to get your final answer of \fbox{$n = 4$}.}
    
    \item {Since the length of a square is equal to its width, we can represent both the length and width with one variable. Let $l$ represent the length/width of the square. If the area of the square is equal to the area of the rectangle, this means that the product of their lengths and widths are equal. This gives us an equation of $l^2$ = 9 x 16. Simplifying the product on the right side of the equation, we get $l^2 = 144$. Finally, we square-root both sides to obtain \fbox{$l = 12$}. (Note: $l = -12$ could also be a solution to the equation, since $(-12)^2 = 144$. However, since the length of the square has to be a positive value, $l = -12$ cannot be a solution.)}
    
    \item
    \begin{enumerate}
        \itemsep2.0em
        \item {One possible solution is \fbox{$x = -2$}.}
        
        \item {Another possible solution is \fbox{$x = -3$}. You could also obtain this answer for Part A and put \fbox{$x = -2$} as your answer here.}
        
        \item {There are three solutions to this equation. They are: \fbox{$x = -2$}, \fbox{$x = -1$}, and \fbox{$x = 1$}.}
        
        \item {From your previous knowledge, you should know that a linear equation (in the case of the question, $n = 1$) has only one unique solution. From Parts (a), (b), and (c), you should know that a quadratic equation ($n = 2$) can have a maximum of two solutions, while a cubic equation ($n = 3$) can have a maximum of three solutions. The pattern here is that the maximum amount of solutions to an equation is equal to its largest exponent ($n$), also known as the degree. Therefore, if $n = 5$, the maximum number of solutions to the equation is equal to \fbox{5}. (Note: a helpful tool to help you visualize these solutions is a graphing calculator, such as Desmos: \url{https://www.desmos.com/calculator} (when you graph the equations, remember to replace the 0 with a $y$). The points on the x-axis where the graph intersects all represent solutions to the equation when it is equal to 0; they are called the roots of the equation. For instance, try graphing $y = x^5 - 15x^4 + 85x^3 - 225x^2 + 274x - 120$ ($n = 5$) to see 5 points lying on the x-axis, each representing a solution to $x^5 - 15x^4 + 85x^3 - 225x^2 + 274x - 120 = 0$.)}
        
    \end{enumerate}
    
\end{enumerate}

\end{document}
